\documentclass[]{article}
\usepackage{lmodern}
\usepackage{amssymb,amsmath}
\usepackage{ifxetex,ifluatex}
\usepackage{fixltx2e} % provides \textsubscript
\ifnum 0\ifxetex 1\fi\ifluatex 1\fi=0 % if pdftex
  \usepackage[T1]{fontenc}
  \usepackage[utf8]{inputenc}
\else % if luatex or xelatex
  \ifxetex
    \usepackage{mathspec}
  \else
    \usepackage{fontspec}
  \fi
  \defaultfontfeatures{Ligatures=TeX,Scale=MatchLowercase}
  \newcommand{\euro}{€}
\fi
% use upquote if available, for straight quotes in verbatim environments
\IfFileExists{upquote.sty}{\usepackage{upquote}}{}
% use microtype if available
\IfFileExists{microtype.sty}{%
\usepackage{microtype}
\UseMicrotypeSet[protrusion]{basicmath} % disable protrusion for tt fonts
}{}
\usepackage{hyperref}
\PassOptionsToPackage{usenames,dvipsnames}{color} % color is loaded by hyperref
\hypersetup{unicode=true,
            pdfborder={0 0 0},
            breaklinks=true}
\urlstyle{same}  % don't use monospace font for urls
\usepackage{longtable,booktabs}
\setlength{\parindent}{0pt}
\setlength{\parskip}{6pt plus 2pt minus 1pt}
\setlength{\emergencystretch}{3em}  % prevent overfull lines
\providecommand{\tightlist}{%
  \setlength{\itemsep}{0pt}\setlength{\parskip}{0pt}}
\setcounter{secnumdepth}{0}

\date{}

% Redefines (sub)paragraphs to behave more like sections
\ifx\paragraph\undefined\else
\let\oldparagraph\paragraph
\renewcommand{\paragraph}[1]{\oldparagraph{#1}\mbox{}}
\fi
\ifx\subparagraph\undefined\else
\let\oldsubparagraph\subparagraph
\renewcommand{\subparagraph}[1]{\oldsubparagraph{#1}\mbox{}}
\fi

\begin{document}

Rumbaugh --Bricker Machine Language Specification Document

Douglas Rumbaugh

Zachary Bricker

Harrisburg University of Science and Technology

Table of Contents

Abstract 6

Conventions 6

Text Formatting Conventions 6

Boldface 6

Emphasis 6

Pseudocode 6

RBML Machine Instructions 6

Hexadecimal 7

The Rumbaugh-Bricker Machine 7

The 32 bit Rumbaugh-Bricker Word 8

Extended Words 9

Register/Memory Access Instructions 10

Store Instructions 10

0x00 STO 10

0x01 STOA 10

0x02 STOJ 11

Load Instructions 11

0x03 LOD 11

0x04 LODA 11

0x05LODJ 12

Move Instructions 12

0x06 MOV 12

0x07 MOVA 12

0x08 MOVJ 13

I/O Instructions 15

Console Read 15

0xB0 CRDM 15

0xB1 CRDR 15

0xB2 CRDA 16

0xB3 CRDJ 16

Console Write 17

0xB4 WRIT 17

0xB5 WRTA 17

0xB6 WRTJ 18

File Read 19

0xB7 FRDM 19

0xB8 FRDR 19

0xB9 FRDA 20

0xBA FRDJ 20

File Write 21

0xBB FWRT 21

0xBC FWTA 21

0xBD FWTJ 22

File Position 22

0xBE REPO 22

0XBF PRES 23

File Access Instructions 23

0xC0 OPEN 23

0xC1 CLOS 23

Arithmetic Instructions 25

Addition 25

0xA0 ADD 25

0XA1 ADDA 25

0xA2 ADDJ 25

Subtraction 26

0xA3 SUB 26

0xA4 SUBA 26

0xA5 SUBJ 26

Division 27

0xA6 DIV 27

0xA7 DIVA 27

0xA8 DIVJ 28

Multiplication 28

0xA9 MULT 28

0xAA MLTA 28

0XAB MLTJ 29

Binary Conjunction 29

0X10 CJNT 29

0x11 CJNA 29

0x12 CJNJ 30

Binary Disjunction 30

0x13 DJNT 30

0x14 DJNA 30

0x15 DJNJ 30

Binary Complement 31

0x16 COMP 31

0x17 CMPA 31

0x18 CMPJ 31

Binary Shift 31

0x19LSFT 31

0x1A LSFA 32

0x1B LSFJ 32

0x1C RSFT 32

0x1D RSFA 32

0x1E RSFJ 33

Modulus 33

0xAC MOD 33

0xAD MODA 33

0xAE MODJ 34

Program Control Instructions 35

Comparison Instructions 35

0x20 CMP 35

0x21 CMPA 35

0x22 CMPJ 36

0x23 CMPM 36

Branch Instructions 36

0x30 BRAN 36

0x31 BRGT 36

0x32 BRLT 37

0x33 BREQ 37

0x34 BRGE 37

0x35 BRLE 37

Call Instructions 37

0x40 CALL 37

0x41 CAGT 37

0x42 CALT 38

0x43 CAEQ 38

0x44 CAGE 38

0x45 CALE 38

0x4F END 39

Halting Instructions 39

0xFF HALT 39

0xEE HERR 39

\textbf{}\\

Abstract

This document contains the basic specification for the Rumbaugh-Bricker
Machine Language instruction (RBML) set, along with suggested
implementations of these instructions in a software emulator for the
machine. RBML is a limited, RISC style instruction set designed to allow
an introduction to machine coding and architecture for students and
hobbyists. It has been boiled down to a minimal set of instructions to
improve accessibility and ease of use. It is not intended to be a fully
featured instruction set, but rather to allow an introduction to the
principles of CPU instruction sets and assembly programming.

Conventions

Text Formatting Conventions

Throughout this document, certain typesetting conventions will be
followed to distinguish important elements within the text. These are as
follows:

Boldface

\textbf{Boldface} text will be used to highlight words of special
meaning in the context of RBML, such as instructions.

Emphasis

Emphasis will be used at the discretion of the authors to bring the
reader's attention to concepts or notes that the authors consider worthy
of special attention or consideration.

Pseudocode

Pseudocode, a highly abstracted language used to express programming
concepts without delving too deeply into specific implementations, is
used throughout this document. Pseudocode will be formatted in the
Courier New typeface, and indented with a line number, as shown here:

if (a .gr. b), then

~ display(a)

else if (a .eq. b), then

~ display(a,b)

else

\begin{itemize}
\tightlist
\item
  ~ display(b)
\end{itemize}

The Pseudocode conventions used in this document should present no
difficulty to readers familiar with the high level languages of FORTRAN
or C. A full definition of the conventions will be provided as an
appendix to final report provided with this project.

RBML Machine Instructions

When presented in form of a coherent operation, perhaps to illustrate
the intended use of an instruction, RBML code, will be written in
Courier New and indented with a line number. There will also be a space
between the instruction and each argument. These spaces are presented
for ease of reading only---and are not an element of the actual
implementation of the machine language by the authors. An example is as
follows:

\begin{itemize}
\tightlist
\item
  + C0 0 0 0006
\end{itemize}

+ C1 0 0 0007

+ A0 0 0 0000

+ B0 0 0 0008

+ E0 0 0 0008

+ FF 0 0 0000

When a numeric literal value is needed, it will be presented without a
separation of groups of bytes, as such:

+ 00000003

+ 00000005

Hexadecimal

Although other variants are certainly possible, the RBML discussed in
this book is for a machine with a 32 bit word length. As it would be
rather cumbersome to spell out each word in the form of 32 binary
digits, hexadecimal will be used as a shorthand throughout this
document. For the uninitiated, hexadecimal is a base 16 counting system,
using the numerals: 0, 1, 2, 3, 4, 5, 6, 7, 8, 9,~ A, B, C, D, E, F. It
maps to binary in a fairly straightforward manner: each hexadecimal
digit represents four binary digits. So the binary number 1111 would be
represented in hexadecimal as F (15 in decimal), and likewise 11111111
would be FF.

As such, there are 2 hexadecimal digits per byte (8 bits)---and 8
hexadecimal digits per word (32 bits). An example of an RBML
instruction, then, could be this:

\begin{itemize}
\tightlist
\item
  0C000013
\end{itemize}

This word would be rendered thus in binary:

\begin{itemize}
\tightlist
\item
  1100000000000000000000010011
\end{itemize}

and in decimal as:

201326611

When pseudocode is used to express operations, and a numeric literal is
used, the base of this literal will be expressed by a prefix to the
number. The standard ANSI C conventions will be used and are as follows:

\begin{longtable}[c]{@{}lll@{}}
\toprule
\begin{minipage}[t]{0.30\columnwidth}\raggedright\strut
Base
\strut\end{minipage} &
\begin{minipage}[t]{0.30\columnwidth}\raggedright\strut
Prefix
\strut\end{minipage} &
\begin{minipage}[t]{0.30\columnwidth}\raggedright\strut
Example
\strut\end{minipage}\tabularnewline
\begin{minipage}[t]{0.30\columnwidth}\raggedright\strut
Decimal
\strut\end{minipage} &
\begin{minipage}[t]{0.30\columnwidth}\raggedright\strut
None
\strut\end{minipage} &
\begin{minipage}[t]{0.30\columnwidth}\raggedright\strut
1045
\strut\end{minipage}\tabularnewline
\begin{minipage}[t]{0.30\columnwidth}\raggedright\strut
Binary
\strut\end{minipage} &
\begin{minipage}[t]{0.30\columnwidth}\raggedright\strut
0b
\strut\end{minipage} &
\begin{minipage}[t]{0.30\columnwidth}\raggedright\strut
0b11011
\strut\end{minipage}\tabularnewline
\begin{minipage}[t]{0.30\columnwidth}\raggedright\strut
Octal
\strut\end{minipage} &
\begin{minipage}[t]{0.30\columnwidth}\raggedright\strut
0
\strut\end{minipage} &
\begin{minipage}[t]{0.30\columnwidth}\raggedright\strut
01761
\strut\end{minipage}\tabularnewline
\begin{minipage}[t]{0.30\columnwidth}\raggedright\strut
Hexadecimal
\strut\end{minipage} &
\begin{minipage}[t]{0.30\columnwidth}\raggedright\strut
0x
\strut\end{minipage} &
\begin{minipage}[t]{0.30\columnwidth}\raggedright\strut
0xFF261
\strut\end{minipage}\tabularnewline
\bottomrule
\end{longtable}

When a number representing a word, an instruction, an argument, or a
memory address appears it will always be expressed in hexadecimal,
unless otherwise noted.

The Rumbaugh-Bricker Machine

Before discussing the language meant to execute in the environment, it
would be ideal to fully define the environment first. The
Rumbaugh-Bricker machine is an idealized computing platform built around
a 32 bit word length. It contains a variable length area of random
access memory, which can be allocated to hold a program, data used by
the program (either pre-defined or written during program execution),
and a call stack (this will be discussed in the section of this document
discussing the `CALL' instruction and its variants). It makes use of 18
registers---R0 through R15, an accumulator, and a jump register. It also
has various CPU flags, which are as follows:

\begin{itemize}
\tightlist
\item
  Greater Than
\item
  Less Than
\item
  Equal To
\item
  Buffer Overrun
\item
  Divide by Zero
\end{itemize}

These flags are set during the execution of CPU instructions. The first
three are comparison flags, set by a comparison type instruction. The
others relate to various error conditions that can occur during
arithmetic operations.

The processor of the Rumbaugh-Bricker machine is a single core, single
threaded CPU incapable of parallel processing. It is capable of
executing only one instruction at a time. The arithmetic-logic unit
(ALU) of the Rumbaugh-Bricker machine is of a simple nature and does not
have built-in support for floating point operations, or complex
mathematical procedures like trigonometric functions. It is fully
capable of the basic arithmetic operations of addition, subtraction,
division, and multiplication, as well as the binary operations of
left/right shift, disjunction, conjunction, and compliment, and also can
perform the modulus operation.~

The Rumbaugh-Bricker machine is capable of receiving input from a
keyboard, and printing input to a console display. It can also read and
write from binary files. There are abstracted instructions in RBML to
allow a program to interface with these sources of input and output
(I/O) that would not exist as instructions on a physical implementation
of the machine. Because of the abstract nature of IO in RBML, these
instructions include indicators specifying whether the treat the data as
numeric, or as characters. The Rumbaugh-Bricker machine has a built in
ASCII decoder for this purpose.

The 32 bit Rumbaugh-Bricker Word

The version of the Rumbaugh-Bricker machine discussed in this document
as a word length of 32 bits. Depending upon the context, these bits will
be divided into various groups. There are two contexts in which a word
can be viewed: as a machine instruction, or as binary data.~

\textbf{F}

\textbf{F}

\textbf{FFFF}

\textbf{FF}

\textbf{Argument 3}

\textbf{Argument 2}

\textbf{Operator}

\textbf{Argument 1}

\textbf{F}

\textbf{F}

\textbf{FFFF}

\textbf{FF}

\textbf{Argument 3}

\textbf{Argument 2}

\textbf{Operator}

\textbf{Argument 1}

~

In most cases, the first two arguments will be used to specify
registers, containing a number between 0 and 15. The third argument is
generally used to specify a memory address---allowing the 32 bit word to
access a maximum of 2 megabytes of memory.

Register/Memory Access Instructions

The Rumbaugh-Bricker machine contains 18 registers and an arbitrary
amount of random access memory. These are the instructions that allow
data to be moved to and from these areas of storage. Every region of
memory, and every register, is defined as containing one 32 bit word.

The instructions that allow access to temporary storage are broken into
three broad categories. These are \textbf{store}, \textbf{load}, and
\textbf{move}. Store instructions allow a word to be moved from a
register and into memory. Load instructions do the reverse---taking data
out of a region of memory and placing it into a register. Finally, move
instructions allow data to be taken from one machine register, and
placed into another.

Store Instructions

0x00 STO

The STO instruction makes use of ARG1 and ARG3 of the word. Any data in
ARG2 will be ignored. STO takes the data contained in the register
defined by ARG1 and places it into the region of memory defined by ARG3.
An example of the use of STO is as follows:

+ 00 2 0 000F

This invocation of the STO instruction will take a word stored in R2,
and move it to the memory address 0x000F.

Any data present in the region of memory defined by ARG3 will be
overwritten by the word in the register defined by ARG1. The value used
in the register defined by ARG1 will remain unchanged. As such, after
the execution of this instruction, the region of memory defined by ARG3
and the register defined by ARG1 should contain the same value.~

This instruction will not set any machine flags, nor alter the machine
state in any way other than the copying of a value from the specified
register to a specified region of memory.

0x01 STOA

The STOA instruction makes use of ARG3 of the word. Any data in ARG1 or
ARG2 will be ignored. STOA takes the data contained in the accumulator
register and places it into the region of memory defined by ARG3. An
example of the use of STOA is as follows:

\begin{itemize}
\tightlist
\item
  + 01 0 0 000F
\end{itemize}

This invocation of the STOA instruction will take a word stored in the
accumulator, and move it to the memory address 0x000F.

Any data present in the region of memory defined by ARG3 will be
overwritten by the word in the accumulator. The value in the accumulator
will remain unchanged. As such, after the execution of this instruction,
the region of memory defined by ARG3 and the accumulator should contain
the same value.

This instruction will not set any machine flags, nor alter the machine
state in any way other than the copying of a value from the specified
register to a specified region of memory.

0x02 STOJ

The STOJ instruction makes use of ARG3 of the word. Any data in ARG1 or
ARG2 will be ignored. STOJ takes the data contained in the jump register
and places it into the region of memory defined by ARG3. An example of
the use of STOJ is as follows:

\begin{itemize}
\tightlist
\item
  + 02 0 0 000F
\end{itemize}

This invocation of the STOJ instruction will take a word stored in the
jump register, and move it to the memory address 0x000F.

Any data present in the region of memory defined by ARG3 will be
overwritten by the word in the jump register. The value in the
accumulator will remain unchanged. As such, after the execution of this
instruction, the region of memory defined by ARG3 and the jump register
should contain the same value.

This instruction will not set any machine flags, nor alter the machine
state in any way other than the copying of a value from the specified
register to a specified region of memory.

The STOJ instruction will be automatically invoked by the machine in the
process of call stack maintenance. For more information on this, refer
to the Rumbaugh-Bricker Machine Specification document.

Load Instructions

0x03 LOD

The LOD instruction makes use of ARG1 and ARG3 of the word. Any data in
ARG2 will be ignored. LOD takes the data stored in the region of memory
specified by ARG3 and places it into the machine register defined by
ARG1. An example of the use of LOAD is as follows:

\begin{itemize}
\tightlist
\item
  + 03 1 0 003E
\end{itemize}

This invocation of the LOD instruction will take a word stored in the
region of memory at 0x003E and place it into R1.

Any data present in the register defined by ARG1 will be overwritten by
the word in the region of memory defined by ARG3. The value in the
region of memory defined by ARG3 will remain unchanged. As such, after
the execution of this instruction, the register defined by ARG1, and the
region of memory defined by R3, should contain the same value.

This instruction will not set any machine flags, nor alter the machine
state in any way other than the copying of a value from the specified
region of memory to the specified register.

0x04 LODA

The LODA instruction makes use of ARG3 of the word. Any data in ARG1 or
ARG2 will be ignored. LODA takes the data stored in the region of memory
specified by ARG3 and places it into the accumulator. An example of the
use of LODA is as follows:

\begin{itemize}
\tightlist
\item
  + 04 0 0 003E
\end{itemize}

This invocation of the LODA instruction will take a word stored in the
region of memory at 0x003E and place it into the accumulator.

Any data present in the accumulator will be overwritten by the word in
the region of memory defined by ARG3. The value in the region of memory
defined by ARG3 will remain unchanged. As such, after the execution of
this instruction, the accumulator, and the region of memory defined by
R3, should contain the same value.

This instruction will not set any machine flags, nor alter the machine
state in any way other than the copying of a value from the specified
region of memory to the specified register.

0x05LODJ

The LODJ instruction makes use of ARG3 of the word. Any data in ARG1 or
ARG2 will be ignored. LODA takes the data stored in the region of memory
specified by ARG3 and places it into the jump register. An example of
the use of LODJ is as follows:

\begin{itemize}
\tightlist
\item
  + 04 0 0 003E
\end{itemize}

This invocation of the LODJ instruction will take a word stored in the
region of memory at 0x003E and place it into the accumulator.

Any data present in the accumulator will be overwritten by the word in
the region of memory defined by ARG3. The value in the region of memory
defined by ARG3 will remain unchanged. As such, after the execution of
this instruction, the accumulator, and the region of memory defined by
R3, should contain the same value.

This instruction will not set any machine flags, nor alter the machine
state in any way other than the copying of a value from the specified
region of memory to the specified register.

The LODJ instruction will be automatically invoked by the machine in the
process of call stack maintenance. For more information on this, refer
to the Rumbaugh-Bricker Machine Specification document.

Move Instructions

0x06 MOV

The MOV instruction makes use of ARG1 and ARG2 of the word. Any data in
ARG3 will be ignored. MOV takes the data stored in a register specified
by ARG1, and places it into a register specified by ARG2. An example of
MOV is as follows:

\begin{itemize}
\tightlist
\item
  + 06 1 3 0000
\end{itemize}

This invocation of the MOV instruction will take a word stored in R1,
and place it into R3.

Any data present in the register defined by ARG2 will be overwritten by
the word in the register defined by ARG1. The data present in the
register defined by ARG1 will remain unchanged. As such, after the
execution of this instruction, the data contained in the register
defined by ARG1 and the data contained in the register defined by ARG2
should be the same.

This instruction will not set any machine flags, nor alter the machine
state in any way other than the copying of data from the specified
register to the specified register.

0x07 MOVA

The MOVA instruction makes use of ARG1 and ARG3 of the word. Any data in
ARG2 will be ignored. MOVA performs three different functions, depending
upon whether the data in ARG3 is equal to zero, or greater than zero. If
the data in ARG3 is greater than zero, then MOVA takes the data stored
in the accumulator and moves it into the register defined by ARG1. If
the data in ARG3 is equal to zero, then MOVA takes the data stored in
the register defined by ARG1 and moves it into the accumulator. An
example of MOVA is as follows:

\begin{itemize}
\tightlist
\item
  + 07 1 0 0000
\end{itemize}

This invocation of the MOVA instruction will take the data stored in R1,
and move it into the accumulator.

When MOVA is used in this manner, any data present in the accumulator
will be overwritten by the word in the register defined by ARG1. The
data present in the register defined by ARG1 will remain unchanged. As
such, after the execution of this instruction, the data contained in the
register defined by ARG1, and the accumulator should be the same.

Another example of the MOVA instruction is as follows:

\begin{itemize}
\tightlist
\item
  + 07 1 0 FFFF
\end{itemize}

Because the content of ARG3 is greater than zero, this invocation of the
MOVA instruction will take the data stored in the accumulator, and move
it into R1.

When MOVA is used in this manner, any data present in the register
defined by ARG1 will be overwritten by the word in the accumulator. The
data present in the accumulator will remain unchanged. As such, after
the execution of this instruction, the data contained in the
accumulator, and the data contained in the register defined by ARG1
should be the same.

This instruction will not set any machine flags, nor alter the machine
state in any way other than the copying of data from the specified
register to the specified register.

0x08 MOVJ

The MOVJ instruction makes use of ARG1 and ARG3 of the word. Any data in
ARG2 will be ignored. MOVJ performs three different functions, depending
upon whether the data in ARG3 is equal to zero, or greater than zero. If
the data in ARG3 is greater than zero, then MOVJ takes the data stored
in the jump register and moves it into the register defined by ARG1. If
the data in ARG3 is equal to zero, then MOVJ takes the data stored in
the register defined by ARG1 and moves it into the jump register. An
example of MOVJ is as follows:

\begin{itemize}
\tightlist
\item
  + 08 1 0 0000
\end{itemize}

This invocation of the MOVJ instruction will take the data stored in R1,
and move it into the jump register.

When MOVJ is used in this manner, any data present in the jump register
will be overwritten by the word in the register defined by ARG1. The
data present in the register defined by ARG1 will remain unchanged. As
such, after the execution of this instruction, the data contained in the
register defined by ARG1, and the jump register should be the same.

Another example of the MOVJ instruction is as follows:

\begin{itemize}
\tightlist
\item
  + 08 1 0 FFFF
\end{itemize}

Because the content of ARG3 is greater than zero, this invocation of the
MOVJ instruction will take the data stored in the jump register, and
move it into R1.

When MOVJ is used in this manner, any data present in the register
defined by ARG1 will be overwritten by the word in the jump register.
The data present in the jump register will remain unchanged. As such,
after the execution of this instruction, the data contained in the jump
register, and the data contained in the register defined by ARG1 should
be the same.

This instruction will not set any machine flags, nor alter the machine
state in any way other than the copying of data from the specified
register to the specified register.

I/O Instructions

Console Read

0xB0 CRDM

The CRDM instruction makes use of ARG1 and ARG3 of the word. Any data in
ARG2 will be ignored. CRDM reads a word from the console and places it
directly into a region of memory specified by ARG3. ARG1 is used to
specify the format of the data being read in. A value of 0 for ARG1
indicates that the data is numeric, and anything else indicates that the
data is a character string. An example of CRDM is as follows:

+ B0 0 0 003F

This invocation of the CRDM instruction will read a numeric word typed
into the console and place it into memory location 0x003F.~

Another example is this:

+ B0 1 0 003F

This invocation of CRDM will read a word's worth of characters (4) from
the console and place them into memory location 0x003F.

In the event that more than one word of data is entered, the machine
overflow flag will be set and only the first word will be read into
memory---the rest of the input will be discarded.

The contents of the region of memory specified by ARG3 will be
overwritten with the word read in from the console.

This instruction will only set the machine overflow flag, and will only
alter the region of memory specified by ARG3. It will not alter the
machine state in any other way.

0xB1 CRDR

The CRDR instruction makes use of ARG1, ARG2 of the word. Any data in
ARG3 will be ignored. CRDR reads a word from the console and places it
into a register specified by ARG2. ARG1 is used to specify the format of
the data being read in. A value of 0 indicates that the data is numeric,
and anything else indicates that the data is a character string. An
example of CRDR is as follows

+ B1 0 4 0000

This invocation of CRDR will read a numeric word typed into the console
and place it into R4.

Another example is this:

+ B1 1 4 0000

This invocation of CRDR will read a word's worth of characters (4) from
the console and place them into R4.

In the event that more than one word of data is entered, the machine
overflow flag will be set, and only the first word will be read into the
register specified by ARG2---the rest of the input will be discarded.

The contents of the register specified by ARG2 will be overwritten with
the word read in from the console.

This instruction will only set the machine overflow flag, and will only
alter the register specified by ARG2. It will not alter the machine
state in any other way.

0xB2 CRDA

The CRDA instruction makes use of ARG1 of the word. Any data in ARG2 or
ARG3 will be ignored. CRDA reads a word from the console and places it
into the accumulator. ARG1 is used to specify the format of the data
being read in. A value of 0 indicates that the data is numeric, and
anything else indicates that the data is a character string. An example
of CRDA is as follows

+ B2 0 0 0000

This invocation of CRDA will read a numeric word typed into the console
and place it into the accumulator.

Another example is this:

+ B2 1 0 0000

This invocation of CRDA will read a word's worth of characters (4) from
the console and place them into the accumulator.

In the event that more than one word of data is entered, the machine
overflow flag will be set, and only the first word will be read into the
accumulator---the rest of the input will be discarded.

The contents of the accumulator will be overwritten with the word read
in from the console.

This instruction will only set the machine overflow flag, and will only
alter the accumulator. It will not alter the machine state in any other
way.

0xB3 CRDJ

The CRDJ instruction makes use of ARG1 of the word. Any data in ARG2 or
ARG3 will be ignored. CRDJ reads a word from the console and places it
into the jump register. ARG1 is used to specify the format of the data
being read in. A value of 0 indicates that the data is numeric, and
anything else indicates that the data is a character string. An example
of CRDJ is as follows

+ B3 0 0 0000

This invocation of CRDJ will read a numeric word typed into the console
and place it into the jump register.

Another example is this:

\begin{itemize}
\tightlist
\item
  + B3 1 0 0000
\end{itemize}

This invocation of CRDJ will read a word's worth of characters (4) from
the console and place them into the jump register.

In the event that more than one word of data is entered, the machine
overflow flag will be set, and only the first word will be read into the
jump register---the rest of the input will be discarded.

The contents of the jump register will be overwritten with the word read
in from the console.

This instruction will only set the machine overflow flag, and will only
alter the jump register. It will not alter the machine state in any
other way.

Console Write

0xB4 WRIT

The WRIT instruction makes use of ARG1 and ARG2 of the word. Any data in
ARG3 will be ignored. WRIT takes a word from the register specified by
ARG2 and writes it to the console. ARG1 is used to specify the format of
the data being written out. A value of 0 indicates that the data is
numeric, and anything else indicates that the data is a character
string.

While writing out character strings, no newlines will be forced by the
system. Each word will be attached end to end in the output, with the
only whitespace being that defined within them. All characters will be
converted from their internal numeric representation to their character
display representation prior to being displayed.

While writing out numeric words, newlines will be forced after each
word. In order to print out multiple numeric words on the same line,
they will need to first be converted into character strings and have
whitespace characters added to them. This is done to allow for multiple
numbers to be printed with subsequent invocations of WRIT, without
needing to alternate between a WRIT for the number, and a WRIT for a
newline or whitespace character. All numbers will be converted to
decimal notation prior to being displayed.

An example of WRIT is as follows:

+ B4 0 4 0000

This invocation of WRIT will print the contents of R4 to the console,
with the data being interpreted as numeric and shown in decimal format.
A newline character will automatically be added by the system at the end
of the number.

Another example of WRIT is this:

+ B4 1 4 0000

This invocation will print the contents of R4 to the console, with the
data being interpreted as a character string and shown in character
format. No additional whitespace characters beyond those contained in
the word will be added by the system at the end of the word.

The contents of the register specified by ARG2 will remain unchanged
after the execution of the WRIT instruction.

This instruction will not set any flags, nor alter the machine state in
any way. Its only purpose is to display data to the console.

0xB5 WRTA

The WRTA instruction makes use of ARG1 of the word. Any data in ARG2 or
ARG3 will be ignored. WRTA takes a word from the accumulator and writes
it to the console. ARG1 is used to specify the format of the data being
written out. A value of 0 indicates that the data is numeric, and
anything else indicates that the data is a character string.

While writing out character strings, no newlines will be forced by the
system. Each word will be attached end to end in the output, with the
only whitespace being that defined within them. All characters will be
converted from their internal numeric representation to their character
display representation prior to being displayed.

While writing out numeric words, newlines will be forced after each
word. In order to print out multiple numeric words on the same line,
they will need to first be converted into character strings and have
whitespace characters added to them. This is done to allow for multiple
numbers to be printed with subsequent invocations of WRTA, without
needing to alternate between a WRTA for the number, and a WRIT for a
newline or whitespace character. All numbers will be converted to
decimal notation prior to being displayed.

An example of WRTA is as follows:

+ B5 0 4 0000

This invocation of WRTA will print the contents of the accumulator to
the console, with the data being interpreted as numeric and shown in
decimal format. A newline character will automatically be added by the
system at the end of the number.

Another example of WRTA is this:

+ B5 1 4 0000

This invocation will print the contents of the accumulator to the
console, with the data being interpreted as a character string and shown
in character format. No additional whitespace characters beyond those
contained in the word will be added by the system at the end of the
word.

The contents of the accumulator will remain unchanged after the
execution of the WRTA instruction.

This instruction will not set any flags, nor alter the machine state in
any way. Its only purpose is to display data to the console.

0xB6 WRTJ

The WRTJ instruction makes use of ARG1 of the word. Any data in ARG2 or
ARG3 will be ignored. WRTJ takes a word from the jump register and
writes it to the console. ARG1 is used to specify the format of the data
being written out. A value of 0 indicates that the data is numeric, and
anything else indicates that the data is a character string.

While writing out character strings, no newlines will be forced by the
system. Each word will be attached end to end in the output, with the
only whitespace being that defined within them. All characters will be
converted from their internal numeric representation to their character
display representation prior to being displayed.

While writing out numeric words, newlines will be forced after each
word. In order to print out multiple numeric words on the same line,
they will need to first be converted into character strings and have
whitespace characters added to them. This is done to allow for multiple
numbers to be printed with subsequent invocations of WRTJ, without
needing to alternate between a WRTJ for the number, and a WRIT for a
newline or whitespace character. All numbers will be converted to
decimal notation prior to being displayed.

An example of WRTJ is as follows:

+ B6 0 4 0000

This invocation of WRTJ will print the contents of the jump register to
the console, with the data being interpreted as numeric and shown in
decimal format. A newline character will automatically be added by the
system at the end of the number.

Another example of WRTJ is this:

+ B6 1 4 0000

This invocation will print the contents of the jump register to the
console, with the data being interpreted as a character string and shown
in character format. No additional whitespace characters beyond those
contained in the word will be added by the system at the end of the
word.

The contents of the jump register will remain unchanged after the
execution of the WRTJ instruction.

This instruction will not set any flags, nor alter the machine state in
any way. Its only purpose is to display data to the console.

File Read

0xB7 FRDM

The FRDM instruction makes use of ARG1 and ARG3 of the word. Any data in
ARG2 will be ignored. FRDM reads the next word in from the open file
specified by ARG1, and places it into the region of memory specified by
ARG3.

FRDM can only read binary files. This does not mean that it cannot read
in character strings---but these strings must be in binary format. A
binary format character string can be displayed as a normal character
string using an appropriately formatted Console Write instruction.

An example of FRDM is this:

+ B7 1 0 046E

This invocation of FRDM will read the next word from File 1, and place
it into the region of memory at 0x046E.

The contents of the file specified by ARG1 will remain unchanged
following the execution of this command, and the contents of the region
of memory defined by ARG3 will be overwritten. As such, the data
contained in the region of memory specified by ARG3, and the contents of
the file specified by ARG1 at the read point in question, should be the
same following the execution of this instruction.

This instruction will not set any flags. It will only alter the region
of memory specified by ARG3. It will not alter the machine state in any
other way.

0xB8 FRDR

The FRDR instruction makes use of ARG1 and ARG2 of the word. Any data in
ARG3 will be ignored. FRDR reads the next word in from the open file
specified by ARG1, and places it into the register specified by ARG2.

FRDR can only read binary files. This does not mean that it cannot read
in character strings---but these strings must be in binary format. A
binary format character string can be displayed as a normal character
string using an appropriately formatted Console Write instruction.

An example of FRDR is this:

\begin{itemize}
\tightlist
\item
  + B8 1 1 0000
\end{itemize}

This invocation of FRDR will read the next word from File 1, and place
it into R1.

The contents of the file specified by ARG1 will remain unchanged
following the execution of this command, and the contents of the
register defined by ARG2 will be overwritten. As such, the data
contained in the register specified by ARG2, and the contents of the
file specified by ARG1 at the read point in question, should be the same
following the execution of this instruction.

This instruction will not set any flags. It will only alter the register
specified by ARG2. It will not alter the machine state in any other way.

0xB9 FRDA

The FRDA instruction makes use of ARG1 of the word. Any data in ARG2 or
ARG3 will be ignored. FRDA reads the next word in from the open file
specified by ARG1, and places it into the accumulator.

FRDA can only read binary files. This does not mean that it cannot read
in character strings---but these strings must be in binary format. A
binary format character string can be displayed as a normal character
string using an appropriately formatted Console Write instruction.

An example of FRDA is this:

+ B9 1 0 0000

This invocation of FRDA will read the next word from File 1, and place
it into the accumulator.

The contents of the file specified by ARG1 will remain unchanged
following the execution of this command, and the contents of the
accumulator will be overwritten. As such, the data contained in the
accumulator, and the contents of the file specified by ARG1 at the read
point in question, should be the same following the execution of this
instruction.

This instruction will not set any flags. It will only alter the
accumulator. It will not alter the machine state in any other way.

0xBA FRDJ

The FRDJ instruction makes use of ARG1 of the word. Any data in ARG2 or
ARG3 will be ignored. FRDJ reads the next word in from the open file
specified by ARG1, and places it into the jump register.

FRDJ can only read binary files. This does not mean that it cannot read
in character strings---but these strings must be in binary format. A
binary format character string can be displayed as a normal character
string using an appropriately formatted Console Write instruction.

An example of FRDJ is this:

+ BA 1 0 0000

This invocation of FRDJ will read the next word from File 1, and place
it into the jump register.

The contents of the file specified by ARG1 will remain unchanged
following the execution of this command, and the contents of the jump
register will be overwritten. As such, the data contained in the jump
register, and the contents of the file specified by ARG1 at the read
point in question, should be the same following the execution of this
instruction.

This instruction will not set any flags. It will only alter the jump
register. It will not alter the machine state in any other way.

File Write

0xBB FWRT

The FWRT instruction makes use of ARG1 and ARG2 of the word. Any data in
ARG3 will be ignored. FWRT writes the word contained in the register
specified by ARG2 to the file specified by ARG1, and advances the write
position in the specified file by one word.

FWRT can only write to binary files. Character strings will be written
using their binary representation. There will be no whitespace or
formatting applied to the file---aside from the ASCII whitespace
characters contained within a written character string.

An example of FRWT is this:

+ BB 1 2 0000

This invocation of FWRT will write the word contained in R2 to File 1,
and advance the write position in File 1 forward one word.

The contents of the register specified by ARG2 will remain unchanged
following the execution of this command, and the contents of the file
specified by ARG1 at the write position will be overwritten. As such,
the data at the initial write position of the file specified by ARG1,
and the data in the register specified by ARG2, should be the same
following the execution of this instruction.

This instruction will not set any machine flags. It will only alter the
file being written to. It will not alter the machine state in any other
way.

0xBC FWTA

The FWTA instruction makes use of ARG1 of the word. Any data in ARG2 or
ARG3 will be ignored. FWTA writes the word contained in the accumulator
to the file specified by ARG1, and advances the write position in the
specified file by one word.

FWTA can only write to binary files. Character strings will be written
using their binary representation. There will be no whitespace or
formatting applied to the file---aside from the ASCII whitespace
characters contained within a written character string.

An example of FWTA is this:

\begin{itemize}
\tightlist
\item
  + BC 1 0 0000
\end{itemize}

This invocation of FWTA will write the word contained in the accumulator
to File 1, and advance the write position in File 1 forward one word.

The contents of the accumulator will remain unchanged following the
execution of this command, and the contents of the file specified by
ARG1 at the write position will be overwritten. As such, the data at the
initial write position of the file specified by ARG1, and the data in
the accumulator, should be the same following the execution of this
instruction.

This instruction will not set any machine flags. It will only alter the
file being written to. It will not alter the machine state in any other
way.

0xBD FWTJ

The FWTJ instruction makes use of ARG1 of the word. Any data in ARG2 or
ARG3 will be ignored. FWTJ writes the word contained in the jump
register to the file specified by ARG1, and advances the write position
in the specified file by one word.

FWTJ can only write to binary files. Character strings will be written
using their binary representation. There will be no whitespace or
formatting applied to the file---aside from the ASCII whitespace
characters contained within a written character string.

An example of FWTJ is this:

+ BD 1 0 0000

This invocation of FWTJ will write the word contained in the jump
register to File 1, and advance the write position in File 1 forward one
word.

The contents of the jump register will remain unchanged following the
execution of this command, and the contents of the file specified by
ARG1 at the write position will be overwritten. As such, the data at the
initial write position of the file specified by ARG1, and the data in
the jump register, should be the same following the execution of this
instruction.

This instruction will not set any machine flags. It will only alter the
file being written to. It will not alter the machine state in any other
way.

File Position

0xBE REPO

The REPO instruction makes use of ARG1, ARG2, and ARG3 of the word. REPO
will advance the write/read position in a file specified by ARG1
forwards or backwards by the number of steps indicated in ARG3. If the
indicated number of steps would move the write/read position past the
beginning or end of the file, the position will stop when the beginning
or end of the file is reached and the remaining steps will be discarded.
The contents of ARG2 will indicate whether to advance forward or
backward. If ARG2 is equal to zero, then the write/read position will be
advanced forwards, towards the end of the file. If ARG2 is greater than
zero, then the write/read position will be advanced backwards, towards
the beginning of the file.

An example of REPO is this:

BE 1 0 0001

This invocation of REPO will advance the write/read position in File 1
forward by one word.

Another example is:

BE 1 1 0001

This invocation of REPO will advance the write/read position in File 1
backwards by one word.

The contents of the file specified by ARG1 will remain unchanged
following the execution of this command.

This instruction will not set any machine flags. It will only alter the
write/read position in the file specified by ARG1. It will not alter the
machine state in any other way.

0XBF PRES

The PRES instruction makes use of ARG1 and ARG2. Any data in ARG3 will
be ignored. PRES will advance the write/read position in a file
specified by ARG1 to either the beginning, or end, of the file,
depending upon the contents of ARG2. If ARG2 contains zero, then the
write/read position is moved to the beginning of the file specified by
ARG1. It ARG2 contains a value greater than zero, then the write/read
position is advanced to the end of the file specified by ARG1.

An example of the PRES instruction is as follows:

BF 1 0 0000

This invocation of PRES will advance the write/read position in File 1
to the beginning of the file. Another example of the instruction is:

BF 1 1 0000

This invocation of PRES will advance the write/read position in File 1
to the end of the file.

The contents of the file specified by ARG1 will remain unchanged
following the execution of this command.

This instruction will not set any machine flags. It will only alter the
write/read position in the file specified by ARG1. It will not alter the
machine state in any other way.

File Access Instructions

0xC0 OPEN

The OPEN instruction makes use of ARG1. Any data in ARG2 or ARG3 will be
ignored. OPEN will make available for use in the program the file
specified by ARG1, and position the write/read position at the beginning
of the file.~

An example of the OPEN instruction is as follows:

C0 3 0 0000

This invocation of OPEN will allow other I/O operations to be performed
upon File 3.

This instruction will not set any machine flags. It will only allow
future I/O instructions to operate upon the file specified by ARG1. It
will not alter the machine state in any other way.

0xC1 CLOS

The CLOS instruction makes use of ARG1. Any data in ARG2 or ARG3 will be
ignored. CLOS will prevent future I/O instructions from accessing the
file specified by ARG1, until it has been opened again using the OPEN
instruction. ~

Executing CLOS on a file will allow its contents to persist on secondary
storage after the end of the program within which a CLOS instruction was
executed.~ If a file was written to, and not properly closed, the data
written to it may not properly persist following the end of the program
in which it was written to.

An example of the CLOS instruction is as follows:

C1 3 0 0000

This invocation of CLOS will prevent future I/O operations from being
performed upon File 3.

This instruction will not set any machine flags. It will only prevent
future I/O instructions from operating upon the file specified by ARG1.
It will not alter the machine state in any other way.

Arithmetic Instructions

Addition

0xA0 ADD

The ADD instruction will make use of ARG1 and ARG2 of the word. Any data
in ARG3 will be ignored. ADD will perform an arithmetic addition upon
the two words in the registers specified by ARG1 and ARG2, and place the
resulting word in the accumulator.

The ADD instruction will be commutative---the order in which the
registers are specified will not affect the result.~

Should the arithmetic addition of the two words result in a value that
is greater than that which can be contained in the accumulator, the
machine overflow flag will be set and the accumulator will be loaded
with the final 32 bits of the result---with the excess leading bits
being truncated.~

The data in the accumulator will be overwritten by the ADD instruction.
It will not affect the contents of any of the other register.~

The ADD instruction will be able to set the overflow flag. It will not
be able to change any other flags. It will not be able to change the
state of the machine in any way other than by changing the contents of
the accumulator and setting the overflow flag.

0XA1 ADDA

The ADDA instruction will make use of ARG1 of the word. Any data in ARG2
or ARG3 will be ignored. ADDA will perform an arithmetic addition upon
the word in the accumulator, and the word in the register specified by
ARG1, and place the resulting word in the accumulator.

Should the arithmetic addition of the two words result in a value that
is greater than that which can be contained in the accumulator, the
machine overflow flag will be set and the accumulator will be loaded
with the final 32 bits of the result---with the excess leading bits
being truncated.

The data in the accumulator will be overwritten by the ADDA instruction.
It will not affect the contents of any of the other register.~

The ADDA instruction will be able to set the overflow flag. It will not
be able to change any other flags. It will not be able to change the
state of the machine in any way other than by changing the contents of
the accumulator and setting the overflow flag.

0xA2 ADDJ

The ADDJ instruction will make use of ARG1 of the word. Any data in ARG2
or ARG3 will be ignored. ADDJ will perform an arithmetic addition upon
the word in the jump register, and the word in the register specified by
ARG1, and place the resulting word in the accumulator.

Should the arithmetic addition of the two words result in a value that
is greater than that which can be contained in the accumulator, the
machine overflow flag will be set and the accumulator will be loaded
with the final 32 bits of the result---with the excess leading bits
being truncated.

The data in the accumulator will be overwritten by the ADDJ instruction.
It will not affect the contents of any of the other register.~

The ADDJ instruction will be able to set the overflow flag. It will not
be able to change any other flags. It will not be able to change the
state of the machine in any way other than by changing the contents of
the accumulator and setting the overflow flag.

Subtraction

0xA3 SUB

The SUB instruction will make use of ARG1 and ARG2 of the word. Any data
in ARG3 will be ignored. SUB will perform an arithmetic subtraction upon
the two words in the registers specified by ARG1 and ARG2, and place the
resulting word in the accumulator.

The SUB instruction will not be commutative---the order of subtraction
will be governed by which argument is used to specify the register. It
will always subtract the register specified by ARG1 from the register
specified by ARG2.

Should the arithmetic subtraction of the two words result in a value
that is smaller than that which can be contained in the accumulator, the
machine overflow flag will be set and the accumulator will be loaded
with the final 32 bits of the result---with the excess leading bits
being truncated.~

The data in the accumulator will be overwritten by the SUB instruction.
It will not affect the contents of any of the other register.~

The SUB instruction will be able to set the overflow flag. It will not
be able to change any other flags. It will not be able to change the
state of the machine in any way other than by changing the contents of
the accumulator and setting the overflow flag

0xA4 SUBA

The SUBA instruction will make use of ARG1 of the word. Any data in ARG2
or ARG3 will be ignored. SUBA will perform an arithmetic subtraction
upon the two words in the registers specified by ARG1 and the
accumulator, and place the resulting word in the accumulator.

The SUBA instruction will not be commutative---the order of subtraction
will be governed by which argument is used to specify the register. It
will always subtract the register specified by ARG1 from the
accumulator.

Should the arithmetic subtraction of the two words result in a value
that is smaller than that which can be contained in the accumulator, the
machine overflow flag will be set and the accumulator will be loaded
with the final 32 bits of the result---with the excess leading bits
being truncated.~

The data in the accumulator will be overwritten by the SUBA instruction.
It will not affect the contents of any of the other register.~

The SUBA instruction will be able to set the overflow flag. It will not
be able to change any other flags. It will not be able to change the
state of the machine in any way other than by changing the contents of
the accumulator and setting the overflow flag.

0xA5 SUBJ

The SUBJ instruction will make use of ARG1 of the word. Any data in ARG2
or ARG3 will be ignored. SUBJ will perform an arithmetic subtraction
upon the two words in the registers specified by ARG1 and the jump
register, and place the resulting word in the accumulator.

The SUBJ instruction will not be commutative---the order of subtraction
will be governed by which argument is used to specify the register. It
will always subtract the register specified by ARG1 from the jump
register.

Should the arithmetic subtraction of the two words result in a value
that is smaller than that which can be contained in the accumulator, the
machine overflow flag will be set and the accumulator will be loaded
with the final 32 bits of the result---with the excess leading bits
being truncated.~

The data in the accumulator will be overwritten by the SUBJ instruction.
It will not affect the contents of any of the other register.~

The SUBJ instruction will be able to set the overflow flag. It will not
be able to change any other flags. It will not be able to change the
state of the machine in any way other than by changing the contents of
the accumulator and setting the overflow flag.

Division

0xA6 DIV

The DIV instruction will make use of ARG1 and ARG2 of the word. Any data
in ARG3 will be ignored. DIV will perform an integer division upon the
two words in the registers specified by ARG1 and ARG2, and place the
resulting word in the accumulator.

The DIV instruction will not be commutative---the order of division will
be governed by which argument is used to specify the register. The
register specified by ARG1 will supply the dividend, and the register
specified by ARG2 will supply the divisor.

Should the word contained in the register defined by ARG2 have a value
of 0, no division will occur. In this case, the divide-by-zero flag will
be set, and the accumulator will be populated with low bits.~

The data in the accumulator will be overwritten by the DIV instruction.
It will not affect the contents of any other register.

The DIV instruction will be able to set the divide-by-zero flag. It will
not be able to change any other flags. It will not be able to change the
state of the machine in any way other than by changing the contents of
the accumulator, and setting the divide-by-zero flag.

0xA7 DIVA

The DIVA instruction will make use of ARG1 of the word. Any data in ARG2
or ARG3 will be ignored. DIVA will perform an integer division upon the
word in the register specified by ARG1 and the word in the accumulator,
and place the resulting word in the accumulator.

The DIVA instruction will not be commutative---the order of division
will be governed by which argument is used to specify the register. The
register specified by ARG1 will supply the dividend, and the accumulator
will supply the divisor.

Should the word contained in the accumulator have a value of 0, no
division will occur. In this case, the divide-by-zero flag will be set,
and the accumulator will be populated with low bits.~

The data in the accumulator will be overwritten by the DIVA instruction.
It will not affect the contents of any other register.

The DIVA instruction will be able to set the divide-by-zero flag. It
will not be able to change any other flags. It will not be able to
change the state of the machine in any way other than by changing the
contents of the accumulator, and setting the divide-by-zero flag.

0xA8 DIVJ

The DIVJ instruction will make use of ARG1 of the word. Any data in ARG2
or ARG3 will be ignored. DIVJ will perform an integer division upon the
word in the register specified by ARG1 and the word in the jump
register, and place the resulting word in the accumulator.

The DIVJ instruction will not be commutative---the order of division
will be governed by which argument is used to specify the register. The
register specified by ARG1 will supply the dividend, and the jump
register will supply the divisor.

Should the word contained in the jump register have a value of 0, no
division will occur. In this case, the divide-by-zero flag will be set,
and the accumulator will be populated with low bits.~

The data in the accumulator will be overwritten by the DIVJ instruction.
It will not affect the contents of any other register.

The DIVJ instruction will be able to set the divide-by-zero flag. It
will not be able to change any other flags. It will not be able to
change the state of the machine in any way other than by changing the
contents of the accumulator, and setting the divide-by-zero flag.

Multiplication

0xA9 MULT

The MULT instruction will make use of ARG1 and ARG2 of the word. Any
data in ARG3 will be ignored. MULT will perform an arithmetic
multiplication of the words in the registers specified by ARG1 and ARG2,
and place the resulting word in the accumulator.

The MULT instruction will be commutative---the order in which the
registers are specified will not affect the result.

Should the product of the multiplication of the words in the registers
defined by ARG1 and ARG2 be too large to contain in the accumulator, the
overflow flag will be set and the accumulator will be loaded with the
last 32 bits of the result. The excess leading bits will be truncated.

The data in the accumulator will be overwritten by the MULT instruction.
It will not affect the contents of any other register.

The MULT instruction will be able to set the overflow flag. It will not
be able to change any other flags. It will not be able to change the
state of the machine in any way other than by changing the contents of
the accumulator, and setting the overflow flag.

0xAA MLTA

The MLTA instruction will make use of ARG1 of the word. Any data in ARG2
or ARG3 will be ignored. MLTA will perform an arithmetic multiplication
of the word in the register specified by ARG1 and the word in the
accumulator, and place the resulting word in the accumulator.

Should the product of the multiplication of the words in the register
defined by ARG1 and the accumulator be too large to contain in the
accumulator, the overflow flag will be set and the accumulator will be
loaded with the last 32 bits of the result. The excess leading bits will
be truncated.

The data in the accumulator will be overwritten by the MLTA instruction.
It will not affect the contents of any other register.

The MLTA instruction will be able to set the overflow flag. It will not
be able to change any other flags. It will not be able to change the
state of the machine in any way other than by changing the contents of
the accumulator, and setting the overflow flag.

0XAB MLTJ

The MLTJ instruction will make use of ARG1 of the word. Any data in ARG2
or ARG3 will be ignored. MLTJ will perform an arithmetic multiplication
of the word in the register specified by ARG1 and the word in the jump
register, and place the resulting word in the accumulator.

Should the product of the multiplication of the words in the register
defined by ARG1 and the jump register be too large to contain in the
accumulator, the overflow flag will be set and the accumulator will be
loaded with the last 32 bits of the result. The excess leading bits will
be truncated.

The data in the accumulator will be overwritten by the MLTJ instruction.
It will not affect the contents of any other register.

The MLTJ instruction will be able to set the overflow flag. It will not
be able to change any other flags. It will not be able to change the
state of the machine in any way other than by changing the contents of
the accumulator, and setting the overflow flag.

Binary Conjunction

0X10 CJNT

The CJNT instruction will make use of ARG1 and ARG2 of the word. Any
data in ARG3 will be ignored. CJNT will perform a binary conjunction
(bitwise and) operation upon the words in the registers defined by ARG1
and ARG2, and place the resulting word in the accumulator.

The CJNT instruction will be commutative---the order in which the
registers are specified will not affect the result.

The data in the accumulator will be overwritten by the CJNT instruction.
It will not affect the contents of any other register.

The CJNT instruction will not be able to set any flags. It will not
affect the machine state in any way other than by changing the contents
of the accumulator.

0x11 CJNA

The CJNA instruction will make use of ARG1 of the word. Any data in ARG2
or ARG3 will be ignored. CJNA will perform a binary conjunction (bitwise
and) operation upon the word in the register defined by ARG1 and the
word in the accumulator, and place the resulting word in the
accumulator.

The data in the accumulator will be overwritten by the CJNA instruction.
It will not affect the contents of any other register.

The CJNA instruction will not be able to set any flags. It will not
affect the machine state in any way other than by changing the contents
of the accumulator.

0x12 CJNJ

The CJNJ instruction will make use of ARG1 of the word. Any data in ARG2
or ARG3 will be ignored. CJNJ will perform a binary conjunction (bitwise
and) operation upon the word in the register defined by ARG1 and the
word in the jump register, and place the resulting word in the
accumulator.

The data in the accumulator will be overwritten by the CJNJ instruction.
It will not affect the contents of any other register.

The CJNJ instruction will not be able to set any flags. It will not
affect the machine state in any way other than by changing the contents
of the accumulator.

Binary Disjunction

0x13 DJNT

The DJNT instruction will make use of ARG1 and ARG2 of the word. Any
data in ARG3 will be ignored. DJNT will perform a binary disjunction
(bitwise or) operation upon the words in the registers defined by ARG1
and ARG2, and place the resulting word in the accumulator.

The DJNT instruction will be commutative---the order in which the
registers are specified will not affect the result.

The data in the accumulator will be overwritten by the DJNT instruction.
It will not affect the contents of any other register.

The DJNT instruction will not be able to set any flags. It will not
affect the machine state in any way other than by changing the contents
of the accumulator.

0x14 DJNA

The DJNA instruction will make use of ARG1 of the word. Any data in ARG2
or ARG3 will be ignored. DJNA will perform a binary disjunction (bitwise
or) operation upon the word in the register defined by ARG1 and the word
in the accumulator, and place the resulting word in the accumulator.

The data in the accumulator will be overwritten by the DJNA instruction.
It will not affect the contents of any other register.

The DJNA instruction will not be able to set any flags. It will not
affect the machine state in any way other than by changing the contents
of the accumulator.

0x15 DJNJ

The DJNJ instruction will make use of ARG1 of the word. Any data in ARG2
or ARG3 will be ignored. DJNJ will perform a binary disjunction (bitwise
or) operation upon the word in the register defined by ARG1 and the word
in the jump register, and place the resulting word in the accumulator.

The data in the accumulator will be overwritten by the DJNJ instruction.
It will not affect the contents of any other register.

The DJNJ instruction will not be able to set any flags. It will not
affect the machine state in any way other than by changing the contents
of the accumulator.

Binary Complement

0x16 COMP

The COMP instruction will make use of ARG1 of the word. Any data in ARG2
or ARG3 will be ignored. COMP will perform a binary complement (bitwise
negation) upon the word in the register specified by ARG1, and place the
resulting word in the accumulator.~

The data in the accumulator will be overwritten by the COMP instruction.
It will not affect any other register.

The COMP instruction will not be able to set any flags. It will not
affect the machine state in any way other than by changing the contents
of the accumulator.

0x17 CMPA

The CMPA instruction will make use of no arguments of the word. Any data
in ARG1, ARG2, or ARG3 will be ignored. CMPA will perform a binary
complement (bitwise negation) upon the word in the accumulator and place
the resulting word in the accumulator.

The data in the accumulator will be overwritten by the CMPA instruction.
It will not affect any other register.

The CMPA instruction will not be able to set any flags. It will not
affect the machine state in any way other than by changing the contents
of the accumulator.

0x18 CMPJ

The CMPJ instruction will make use of no arguments of the word. Any data
in ARG1, ARG2, or ARG3 will be ignored. CMPJ will perform a binary
complement (bitwise negation) upon the word in the jump register and
place the resulting word in the accumulator.

The data in the accumulator will be overwritten by the CMPJ instruction.
It will not affect any other register.

The CMPJ instruction will not be able to set any flags. It will not
affect the machine state in any way other than by changing the contents
of the accumulator.

Binary Shift

0x19LSFT

The LSFT instruction will make use of ARG1 and ARG2 of the word. Any
data in ARG3 will be ignored. LSFT will perform a left bit-shift on the
word contained in the register specified by ARG1 a number of times equal
to the value of the word contained in the register specified by ARG2,
and place the resulting word in the accumulator.

The data in the accumulator will be overwritten by the LSFT instruction.
It will not affect the contents of any other register.

The LSFT instruction will not set any flags. It will not be able to
change the machine state in any way other than by changing the contents
of the accumulator.

0x1A LSFA

The LSFA instruction will make use of ARG1 of the word. Any data in ARG2
or ARG3 will be ignored. LSFA will perform a left bit-shift on the word
contained in the register specified by ARG1 a number of times equal to
the value of the word contained in the accumulator, and place the
resulting word in the accumulator.

The data in the accumulator will be overwritten by the LSFA instruction.
It will not affect the contents of any other register.

The LSFA instruction will not set any flags. It will not be able to
change the machine state in any way other than by changing the contents
of the accumulator.

0x1B LSFJ

The LSFJ instruction will make use of ARG1 of the word. Any data in ARG2
or ARG3 will be ignored. LSFJ will perform a left bit-shift on the word
contained in the register specified by ARG1 a number of times equal to
the value of the word contained in the jump register, and place the
resulting word in the accumulator.

The data in the accumulator will be overwritten by the LSFJ instruction.
It will not affect the contents of any other register.

The LSFJ instruction will not set any flags. It will not be able to
change the machine state in any way other than by changing the contents
of the accumulator.

0x1C RSFT

The RSFT instruction will make use of ARG1 and ARG2 of the word. Any
data in ARG3 will be ignored. RSFT will perform a right bit-shift on the
word contained in the register specified by ARG1 a number of times equal
to the value of the word contained in the register specified by ARG2,
and place the resulting word in the accumulator.

The data in the accumulator will be overwritten by the RSFT instruction.
It will not affect the contents of any other register.

The RSFT instruction will not set any flags. It will not be able to
change the machine state in any way other than by changing the contents
of the accumulator.

0x1D RSFA

The RSFA instruction will make use of ARG1 of the word. Any data in ARG2
or ARG3 will be ignored. RSFA will perform a right bit-shift on the word
contained in the register specified by ARG1 a number of times equal to
the value of the word contained in the accumulator, and place the
resulting word in the accumulator.

The data in the accumulator will be overwritten by the RSFA instruction.
It will not affect the contents of any other register.

The RSFA instruction will not set any flags. It will not be able to
change the machine state in any way other than by changing the contents
of the accumulator.

0x1E RSFJ

The RSFJ instruction will make use of ARG1 of the word. Any data in ARG2
or ARG3 will be ignored. RSFJ will perform a right bit-shift on the word
contained in the register specified by ARG1 a number of times equal to
the value of the word contained in the jump register, and place the
resulting word in the accumulator.

The data in the accumulator will be overwritten by the RSFJ instruction.
It will not affect the contents of any other register.

The RSFJ instruction will not set any flags. It will not be able to
change the machine state in any way other than by changing the contents
of the accumulator.

Modulus

0xAC MOD

The MOD instruction will make use of ARG1 and ARG2 of the word. Any data
in ARG3 will be ignored. MOD will perform a modular division upon the
two words in the registers specified by ARG1 and ARG2, and place the
resulting word in the accumulator.

The MOD instruction will not be commutative---the order of division will
be governed by which argument is used to specify the register. The
register specified by ARG1 will supply the dividend, and the register
specified by ARG2 will supply the divisor.

Should the word contained in the register defined by ARG2 have a value
of 0, no division will occur. In this case, the divide-by-zero flag will
be set, and the accumulator will be populated with low bits.~

The data in the accumulator will be overwritten by the MOD instruction.
It will not affect the contents of any other register.

The MOD instruction will be able to set the divide-by-zero flag. It will
not be able to change any other flags. It will not be able to change the
state of the machine in any way other than by changing the contents of
the accumulator, and setting the divide-by-zero flag.

0xAD MODA

The MODA instruction will make use of ARG1 of the word. Any data in ARG2
or ARG3 will be ignored. MODA will perform a modular division upon the
word in the register specified by ARG1 and the word in the accumulator,
and place the resulting word in the accumulator.

The MODA instruction will not be commutative---the order of division
will be governed by which argument is used to specify the register. The
register specified by ARG1 will supply the dividend, and the accumulator
will supply the divisor.

Should the word contained in the accumulator have a value of 0, no
division will occur. In this case, the divide-by-zero flag will be set,
and the accumulator will be populated with low bits.~

The data in the accumulator will be overwritten by the MODA instruction.
It will not affect the contents of any other register.

The MODA instruction will be able to set the divide-by-zero flag. It
will not be able to change any other flags. It will not be able to
change the state of the machine in any way other than by changing the
contents of the accumulator, and setting the divide-by-zero flag.

0xAE MODJ

The MODJ instruction will make use of ARG1 of the word. Any data in ARG2
or ARG3 will be ignored. MODJ will perform a modular division upon the
word in the register specified by ARG1 and the word in the jump
register, and place the resulting word in the accumulator.

The MODJ instruction will not be commutative---the order of division
will be governed by which argument is used to specify the register. The
register specified by ARG1 will supply the dividend, and the jump
register will supply the divisor.

Should the word contained in the jump register have a value of 0, no
division will occur. In this case, the divide-by-zero flag will be set,
and the accumulator will be populated with low bits.~

The data in the accumulator will be overwritten by the MODJ instruction.
It will not affect the contents of any other register.

The MODJ instruction will be able to set the divide-by-zero flag. It
will not be able to change any other flags. It will not be able to
change the state of the machine in any way other than by changing the
contents of the accumulator, and setting the divide-by-zero flag.

Program Control Instructions

RBML contains many instructions dedicated to the process control of a
program. These instructions allow for branches or calls to other regions
of memory, as well as numeric comparisons. The comparison operations set
the machine comparison flags, which can then be used by conditional
branch or call instructions to allow for the creation of loops,
branches, and conditional subroutine calls. The halting instructions can
also be considered program control.

Comparison Instructions

0x20 CMP

CMP will make use of ARG1 and ARG2 of the word. Any data in ARG3 will be
ignored. CMP will compare the contents of the word in the register
specified by ARG1 and the contents of the word in the register specified
by ARG2 and update the comparison flags according to the following
rules:

\begin{itemize}
\tightlist
\item
  If the content of the register specified by ARG1 is greater than the
  contents of the register specified by ARG2, set the greater-than flag.
\item
  If the content of the register specified by ARG1 is equal to the
  contents of the register specified by ARG2, set the equal-to flag.
\item
  If the content of the register specified by ARG1 is less than the
  contents of the register specified by ARG2, set the less-than flag.
\end{itemize}

The CMP instruction will not alter the contents of any machine registers
in any way.

Before this instruction sets any of the comparison flags---it will unset
them all. It will then set one, and only one, flag according to the
rules listed above. It will not alter the machine state in any other
way.

0x21 CMPA

CMPA will make use of ARG1 of the word. Any data in ARG2 or ARG3 will be
ignored. CMPA will compare the contents of the word in the accumulator
and the contents of the word in the register specified by ARG1 and
update the comparison flags according to the following rules:

\begin{itemize}
\tightlist
\item
  If the content of the accumulator is greater than the contents of the
  register specified by ARG1, set the greater-than flag.
\item
  If the content of the accumulator is equal to the contents of the
  register specified by ARG1, set the equal-to flag.
\item
  If the content of the accumulator is less than the contents of the
  register specified by ARG1, set the less-than flag.
\end{itemize}

The CMPA instruction will not alter the contents of any machine
registers in any way.

Before this instruction sets any of the comparison flags---it will unset
them all. It will then set one, and only one, flag according to the
rules listed above. It will not alter the machine state in any other
way.

0x22 CMPJ

CMPJ will make use of ARG1 of the word. Any data in ARG2 or ARG3 will be
ignored. CMPJ will compare the contents of the word in the accumulator
and the contents of the word in the register specified by ARG1 and
update the comparison flags according to the following rules:

\begin{itemize}
\tightlist
\item
  If the content of the jump register is greater than the contents of
  the register specified by ARG1, set the greater-than flag.
\item
  If the content of the jump register is equal to the contents of the
  register specified by ARG1, set the equal-to flag.
\item
  If the content of the jump register is less than the contents of the
  register specified by ARG1, set the less-than flag.
\end{itemize}

The CMPJ instruction will not alter the contents of any machine
registers in any way.

Before this instruction sets any of the comparison flags---it will unset
them all. It will then set one, and only one, flag according to the
rules listed above. It will not alter the machine state in any other
way.

0x23 CMPM

CMPM will make use of ARG1 and ARG3 of the word. Any data in ARG2 will
be ignored. CMPM will compare the contents of the word in the register
specified by ARG1 with the contents of the word in the region of memory
specified by ARG3, and update the comparison flags according to the
following rules:

\begin{itemize}
\tightlist
\item
  If the content of the register specified by ARG1 is greater than the
  contents of region of memory specified by ARG3, set the greater-than
  flag.
\item
  If the content of the register specified by ARG1 is equal to the
  contents of region of memory specified by ARG3, set the equal-to flag.
\item
  If the content of the register specified by ARG1 is less than the
  contents of region of memory specified by ARG3, set the less-than
  flag.
\end{itemize}

The CMPM instruction will not alter the contents of any machine
registers in any way.

Before this instruction sets any of the comparison flags---it will unset
them all. It will then set one, and only one, flag according to the
rules listed above. It will not alter the machine state in any other
way.

Branch Instructions

0x30 BRAN

The BRAN instruction will make use of ARG3 of the word. Any data in ARG1
or ARG2 will be ignored. BRAN will set the instruction counter equal to
the content of ARG3.

The BRAN instruction will not set any flags, or alter the contents of
any register. It will not alter the machine state in any way other than
the adjustment of the instruction counter.

0x31 BRGT

The BRGT instruction will make use of ARG3 of the word. Any data in ARG1
or ARG2 will be ignored. BRGT will set the instruction counter equal to
the content of ARG3, if and only if the greater-than flag is set.~

The BRGT instruction will not set any flags, or alter the contents of
any register. It will not alter the machine state in any way other than
the adjustment of the instruction counter.

0x32 BRLT

The BRLT instruction will make use of ARG3 of the word. Any data in ARG1
or ARG2 will be ignored. BRLT will set the instruction counter equal to
the content of ARG3, if and only if the less-than flag is set.~

The BRLT instruction will not set any flags, or alter the contents of
any register. It will not alter the machine state in any way other than
the adjustment of the instruction counter.

0x33 BREQ

The BREQ instruction will make use of ARG3 of the word. Any data in ARG1
or ARG2 will be ignored. BREQ will set the instruction counter equal to
the content of ARG3, if and only if the equal-to flag is set.~

The BREQ instruction will not set any flags, or alter the contents of
any register. It will not alter the machine state in any way other than
the adjustment of the instruction counter.

0x34 BRGE

The BRGE instruction will make use of ARG3 of the word. Any data in ARG1
or ARG2 will be ignored. BRGE will set the instruction counter equal to
the content of ARG3, if and only if the greater-than flag is set, or the
equal to flag is set.~

The BRGE instruction will not set any flags, or alter the contents of
any register. It will not alter the machine state in any way other than
the adjustment of the instruction counter.

0x35 BRLE

The BRLE instruction will make use of ARG3 of the word. Any data in ARG1
or ARG2 will be ignored. BRLE will set the instruction counter equal to
the content of ARG3, if and only if the less-than flag is set, or the
equal to flag is set.~

The BRLE instruction will not set any flags, or alter the contents of
any register. It will not alter the machine state in any way other than
the adjustment of the instruction counter

0x36 BRER

The BRER instruction will make use of ARG3 of the word. Any data in ARG1
or ARG2 will be ignored. BRER will set the instruction counter equal to
the content of ARG3, if and only if the machine overflow flag, machine
divide-by-zero flag, or the machine IO error flag is set.

The BRER instruction will not set any flags, or alter the contents of
any register. It will not alter the machine state in any way other than
the adjustment of the instruction counter

0x37 BROF

The BROF instruction will make use of ARG3 of the word. Any data in ARG1
or ARG2 will be ignored. BROF will set the instruction counter equal to
the content of ARG3, if and only if the machine overflow flag is set.~

The BROF instruction will not set any flags, or alter the contents of
any register. It will not alter the machine state in any way other than
the adjustment of the instruction counter

0x38 BRIO

The BRIO instruction will make use of ARG3 of the word. Any data in ARG1
or ARG2 will be ignored. BRIO will set the instruction counter equal to
the content of ARG3, if and only if the machine IO error flag is set.~

The BRIO instruction will not set any flags, or alter the contents of
any register. It will not alter the machine state in any way other than
the adjustment of the instruction counter

0x39 BRDZ

The BRDZ instruction will make use of ARG3 of the word. Any data in ARG1
or ARG2 will be ignored. BRDZ will set the instruction counter equal to
the content of ARG3, if and only if the machine divide-by-zero flag is
set.~

The BRDZ instruction will not set any flags, or alter the contents of
any register. It will not alter the machine state in any way other than
the adjustment of the instruction counter

Call Instructions

0x40 CALL

The CALL instruction will make use of ARG3 of the word. Any data in ARG1
or ARG2 will be ignored. CALL will set the jump register equal to the
instruction counter. If the jump register contains anything other than
high bits, it will first push the current value of the jump register to
the top of the call stack. Once the jump register contains the value of
the instruction counter, the instruction counter will be overwritten to
contain the content of ARG3.

The CALL instruction will not set any flags, nor alter the contents of
any register other than the jump register. It will not alter the machine
state in any way other than the adjustment of the instruction counter,
jump register, and pushing words to the call stack.

0x41 CAGT

The CAGT instruction will make use of ARG3 of the word. Any data in ARG1
or ARG2 will be ignored. If and only if the greater-than flag is set,
CAGT will set the jump register equal to the instruction counter. If the
jump register contains anything other than high bits, it will first push
the current value of the jump register to the top of the call stack.
Once the jump register contains the value of the instruction counter,
the instruction counter will be overwritten to contain the content of
ARG3.

The CAGT instruction will not set any flags, nor alter the contents of
any register other than the jump register. It will not alter the machine
state in any way other than the adjustment of the instruction counter,
jump register, and pushing words to the call stack.

0x42 CALT

The CALT instruction will make use of ARG3 of the word. Any data in ARG1
or ARG2 will be ignored. If and only if the less-than flag is set, CALT
will set the jump register equal to the instruction counter. If the jump
register contains anything other than high bits, it will first push the
current value of the jump register to the top of the call stack. Once
the jump register contains the value of the instruction counter, the
instruction counter will be overwritten to contain the content of ARG3.

The CALT instruction will not set any flags, nor alter the contents of
any register other than the jump register. It will not alter the machine
state in any way other than the adjustment of the instruction counter,
jump register, and pushing words to the call stack.

0x43 CAEQ

The CAEQ instruction will make use of ARG3 of the word. Any data in ARG1
or ARG2 will be ignored. If and only if the equal-to flag is set, CAEQ
will set the jump register equal to the instruction counter. If the jump
register contains anything other than high bits, it will first push the
current value of the jump register to the top of the call stack. Once
the jump register contains the value of the instruction counter, the
instruction counter will be overwritten to contain the content of ARG3.

The CAEQ instruction will not set any flags, nor alter the contents of
any register other than the jump register. It will not alter the machine
state in any way other than the adjustment of the instruction counter,
jump register, and pushing words to the call stack.

0x44 CAGE

The CAGE instruction will make use of ARG3 of the word. Any data in ARG1
or ARG2 will be ignored. If and only if the greater-than flag or the
equal-to flag is set, CAGE will set the jump register equal to the
instruction counter. If the jump register contains anything other than
high bits, it will first push the current value of the jump register to
the top of the call stack. Once the jump register contains the value of
the instruction counter, the instruction counter will be overwritten to
contain the content of ARG3.

The CAGE instruction will not set any flags, nor alter the contents of
any register other than the jump register. It will not alter the machine
state in any way other than the adjustment of the instruction counter,
jump register, and pushing words to the call stack.

0x45 CALE

The CALE instruction will make use of ARG3 of the word. Any data in ARG1
or ARG2 will be ignored. If and only if the less-than flag or the equal
to flag is set, CALE will set the jump register equal to the instruction
counter. If the jump register contains anything other than high bits, it
will first push the current value of the jump register to the top of the
call stack. Once the jump register contains the value of the instruction
counter, the instruction counter will be overwritten to contain the
content of ARG3.

The CALE instruction will not set any flags, nor alter the contents of
any register other than the jump register. It will not alter the machine
state in any way other than the adjustment of the instruction counter,
jump register, and pushing words to the call stack.

0x46 CAER

The CAER instruction will make use of ARG3 of the word. Any data in ARG1
or ARG2 will be ignored. If and only if the machine overflow, machine
divide-by-zero, or the machine IO error flag is set, CAER will set the
jump register equal to the instruction counter. If the jump register
contains anything other than high bits, it will first push the current
value of the jump register to the top of the call stack. Once the jump
register contains the value of the instruction counter, the instruction
counter will be overwritten to contain the content of ARG3.

The CAER instruction will not set any flags, nor alter the contents of
any register other than the jump register. It will not alter the machine
state in any way other than the adjustment of the instruction counter,
jump register, and pushing words to the call stack.

0x47 CAOF

The CAOF instruction will make use of ARG3 of the word. Any data in ARG1
or ARG2 will be ignored. If and only if the machine overflow flag is
set, CAOF will set the jump register equal to the instruction counter.
If the jump register contains anything other than high bits, it will
first push the current value of the jump register to the top of the call
stack. Once the jump register contains the value of the instruction
counter, the instruction counter will be overwritten to contain the
content of ARG3.

The CAOF instruction will not set any flags, nor alter the contents of
any register other than the jump register. It will not alter the machine
state in any way other than the adjustment of the instruction counter,
jump register, and pushing words to the call stack.

0x48 CAIO

The CAIO instruction will make use of ARG3 of the word. Any data in ARG1
or ARG2 will be ignored. If and only if the machine IO error flag is
set, CAIO will set the jump register equal to the instruction counter.
If the jump register contains anything other than high bits, it will
first push the current value of the jump register to the top of the call
stack. Once the jump register contains the value of the instruction
counter, the instruction counter will be overwritten to contain the
content of ARG3.

The CAIO instruction will not set any flags, nor alter the contents of
any register other than the jump register. It will not alter the machine
state in any way other than the adjustment of the instruction counter,
jump register, and pushing words to the call stack.

0x49 CADZ

The CADZ instruction will make use of ARG3 of the word. Any data in ARG1
or ARG2 will be ignored. If and only if the machine divide-by-zero flag
is set, CADZ will set the jump register equal to the instruction
counter. If the jump register contains anything other than high bits, it
will first push the current value of the jump register to the top of the
call stack. Once the jump register contains the value of the instruction
counter, the instruction counter will be overwritten to contain the
content of ARG3.

The CADZ instruction will not set any flags, nor alter the contents of
any register other than the jump register. It will not alter the machine
state in any way other than the adjustment of the instruction counter,
jump register, and pushing words to the call stack.

0x4F END

The END instruction will not make use of any arguments of the word. Any
data in ARG1, ARG2, or ARG3 will be ignored. END will set the
instruction counter equal to the contents of the jump register, and then
pop the top word from the call stack and place it into the jump
register. If the call stack is empty, END will populate the jump
register with high bits.

The END instruction will not set any flags, nor alter the contents of
any register other than the jump register. It will not alter the machine
state in any way other than the adjustment of the instruction counter,
jump register, and popping words from the call stack.

Halting Instructions

0xFF HALT

The HALT instruction will not make use of any argument of the word. Any
data in ARG1, ARG2, or ARG3 will be ignored. HALT will case an immediate
end to machine execution.

0xEE HERR

The HERR instruction will not make use of any argument of the word. Any
data in ARG1, ARG2, or ARG3 will be ignored. HERR will cause an
immediate end to machine execution, and initiate a dump of all details
of the machine state to an error file, and to the display.

\end{document}
